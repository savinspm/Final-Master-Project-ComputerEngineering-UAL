
%----------------------------------------------------------------------------------------
%	CHAPTER 3
%----------------------------------------------------------------------------------------


\chapterimage{dibujo} % Chapter heading image

\chapter{Cap�tulo 3 }\label{chapter_3}
\section{Seccion 1}
\subsection{Figura posicionada en lateral}
\pichskip{2em}
\parpic[r][t]{%
	\begin{minipage}{40mm}
		\includegraphics[width=\linewidth]{example-image-a}%
	\end{minipage}
}
\subsubsection{Subsecci�n 1}
\lipsum[1]

\parpic[r][t]{%
	\begin{minipage}{40mm}
		\includegraphics[width=\linewidth]{example-image-b}%
	\end{minipage}
}
{ \subsubsection{Subseccion 2}}

\lipsum[1]

\begin{minipage}{\textwidth}
\lstinputlisting[language=C,label=example-CodeC, caption={Fragmento del codigo desde archivo.}]{codigoFuente/code.c}
\end{minipage}

\begin{minipage}{\textwidth}
\begin{lstlisting}[language=C,label=example_Code, caption={Fragmento del codigo pegado directamente.}]
#include <stdio.h>
#define N 10
/* Comentario en un
* bloque*/

int main()
{
	int i;

	// Line comment.	
	puts("�Hola Mundo!");

	for (i = 0; i < N; i++)
	{
		puts("�LaTeX tambi�n es genial para los programadores!");
	}
	
return 0;
}
\end{lstlisting}
\end{minipage}

\subsubsection{Figuras en paralelo}


\begin{figure}[!htb]
	\centering
	\begin{minipage}{0.49\textwidth}
		\centering
		\includegraphics[width=\textwidth]{example-image-c}
		\caption{Figura c.}
		\label{img_figc}
	\end{minipage}
	~
	\begin{minipage}{0.49\textwidth}
		\centering
		\includegraphics[width=\textwidth]{example-image-a}
		\caption{Figura a.}
		\label{img_figd}
	\end{minipage}
	
\end{figure}

